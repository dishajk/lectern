\documentclass[12pt,a4paper]{article}
\usepackage[none]{hyphenat}
\title{How Folding Paper Creates Mathematics}
\author{Disha Kuzhively}
\begin{document}
\maketitle
\begin{abstract}
    Mathematics is often seen as something that lives on a page, written with symbols and equations. In this workshop, we will instead explore mathematics through the art of paper folding. In particular, we will examine criteria for flat-foldability, that is, the conditions under which a given crease pattern can be folded flat and pressed inside a book without crumpling. Through hands-on folding, we will formulate conjectures, test their validity, and prove some of them together. No advanced mathematical background or prior experience with origami is required. All that is needed is curiosity and a willingness to fold paper. Participants will create an origami model and take home not only the folded model, but also the mathematical ideas that emerge from the folding process.
\end{abstract}
\end{document}