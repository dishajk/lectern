\documentclass[12pt,a4paper]{article}
\usepackage[none]{hyphenat}
\title{Fractals}
\author{Disha Kuzhively}
\begin{document}
\maketitle
\begin{abstract}
    In the last session, we explored the symmetries we see around us and studied two-dimensional patterns that create beautiful repeating designs. This time, we will push the idea further and ask a few questions: Is it possible to tile an entire flat surface using shapes that do not repeat regularly? In other words, can we cover the plane without any repeating pattern at all? For a long time, mathematicians wondered about this. In 2023, David Smith, Joseph Samuel Myers, Craig S. Kaplan, and Chaim Goodman-Strauss improved the answer to this question by discovering an aperiodic monotile - a single shape that can tile the plane, but only in a non-repeating way.
    
    In this session, we will play with these tiles, experiment with making our own tilings, and see what makes them so special. We will also take a short look at fractals (shapes that show repeating structure at different scales) and compare these with the non-repeating patterns made by aperiodic tiles.
    
    The workshop will be hands-on and exploratory, with no advanced mathematics required just curiosity and a willingness to try things out.
\end{abstract}
\end{document}